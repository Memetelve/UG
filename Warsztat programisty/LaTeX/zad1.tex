\documentclass[a4paper,12pt]{article}

\usepackage[utf8]{inputenc}
\usepackage[T1]{fontenc}
\usepackage{amsmath}
\usepackage{graphicx}
\usepackage{booktabs}

\title{Tytuł pracy}
\author{Imię i nazwisko}
\date{\today}

\begin{document}

\maketitle

\begin{abstract}
Streszczenie pracy.
\end{abstract}

\section{Pierwszy rozdział}

\subsection{Podsekcja 1}

Tekst.

\subsection{Podsekcja 2}

Tekst.

\section{Drugi rozdział}

\subsection{Podsekcja 1}

Tekst.

\subsection{Podsekcja 2}

Tekst.

\subsection{Podsekcja 3}

Tekst.

\begin{figure}[h]
\centering
\includegraphics[width=0.7\textwidth]{rysunek1.pdf}
\caption{Opis rysunku 1}
\end{figure}

\begin{figure}[h]
\centering
\includegraphics[width=0.7\textwidth]{rysunek2.pdf}
\caption{Opis rysunku 2}
\end{figure}

\begin{table}[h]
\centering
\begin{tabular}{@{}ccc@{}}
\toprule
Kolumna 1 & Kolumna 2 & Kolumna 3 \ \midrule
Wartość 1 & Wartość 2 & Wartość 3 \
Wartość 4 & Wartość 5 & Wartość 6 \ \bottomrule
\end{tabular}
\caption{Opis tabeli}
\end{table}

W tekście odwołujemy się do rysunków (np. na rysunku 1 widzimy...), jak i do tabeli (tabela 1 przedstawia...).

Bibliografia:

\begin{thebibliography}{9}
\bibitem{bib1} Autor 1, Tytuł 1, Rok 1.
\bibitem{bib2} Autor 2, Tytuł 2, Rok 2.
\end{thebibliography}

W tekście odwołujemy się do bibliografii (np. według [1]...).

\end{document}
